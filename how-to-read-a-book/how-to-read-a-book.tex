\documentclass[a4paper,11pt]{scrartcl}

\usepackage[ngerman]{babel}
\usepackage{apacite}

\usepackage{chngcntr}
    \counterwithout{subsection}{section}

\usepackage{fontspec}
    \setmainfont{EB Garamond}
    \setsansfont{Open Sans}

\usepackage{setspace}
    \onehalfspace

\begin{document}

\title{How to Read a Book\\\cite{adler-vandooren}}
\subtitle{Zusammenfassung}
\author{Patrick Bucher}
\date{\today}
\maketitle

\tableofcontents

\newpage
\section*{Teil I: Die Dimensionen des Lesens}
\addcontentsline{toc}{section}{Teil I: Die Dimensionen des Lesens}

\subsection{Die Aktivität und Kunst des Lesens}

Wie ein Buch mehr oder weniger aktiv gelesen werden kann, so kann es auch mehr oder weniger gut verstanden werden. Ein guter Leser schöpft mehr Verständnis – und nicht nur Informationen – aus einem Buch als ein schlechter Leser. Zwar kann ein Leser ein Buch zum reinen Informationsgewinn lesen, sofern er dessen Inhalt auf Anhieb vollständig versteht. Dazu müssen Leser und Autor was ihr Verständnis betrifft auf gleicher Stufe stehen. Nur wenn der Leser ein Buch auf Anhieb nicht vollständig versteht, kann dessen Lektüre zu seinem Verständnis beitragen. Zur Erlangung des zusätzlichen Verständnisses kann der Leser entweder ein anderes Buch oder eine andere Person zu Rate ziehen – oder aber sich ohne die Anleitung eines anderen, also durch reines Nachdenken, auf die gleiche Verständnisstufe mit dem Autor bringen. Dies bezeichnet man als gutes Lesen. Je besser ein Leser ist, desto mehr Verständnis kann er aus einem ihm bis dato schwer verständlichen Buch schöpfen, sofern es sich dabei um ein gutes Buch handelt – ein Buch, das ihm neues Verständnis bringen kann.

\subsection{Die Stufen des Lesens}

Es gibt vier Stufen des Lesens, wobei jede Stufe die jeweils tiefere(n) Stufe(n) beinhaltet. Bei der ersten Stufe, dem elementaren Lesen, welches in der Schule eingeübt wird, geht es um das Entziffern von Wörtern und Sätzen. Auf der zweiten Stufe, dem inspizierenden Lesen, geht es darum, sich in kürzester Zeit einen Überblick über ein Buch zu verschaffen. Die dritte Stufe, das analytische Lesen, umfasst das Durchlesen des Buches zur Aufnahme dessen Inhalts. Gründliches Verständnis ist hierbei das Ziel, darum kann für diese Stufe beliebig viel Zeit aufgewendet werden. Auf der vierten Stufe, dem syntopischen Lesen, werden mehrere Bücher eines Themengebiets vergleichend gelesen, um so zu einem höheren, über den einzelnen Büchern stehenden Verständnis zu gelangen. Geübte Leser führen das inspizierende Lesen (Stufe zwei) und das analytische Lesen (Stufe drei) im gleichen Arbeitsschritt durch.

\subsection{Die erste Stufe des Lesens: Elementares Lesen}

Schüler werden nur während der ersten sechs bis neun Schuljahre im Lesen unterwiesen. Sie sind dann funktional lesefähig. Dies bedeutet, dass sie zum Zweck der Informationsaufnahme lesen, aber nicht, dass sie viel Verständnis aus anspruchsvollen Büchern schöpfen können. Darum muss das Lesen, das über die elementare Stufe hinausgeht, auch nach den Pflichtschuljahren weiter eingeübt werden.

\subsection{Die zweite Stufe des Lesens: Inspizierendes Lesen}

Im inspizierenden Lesen geht es darum, in wenigen Minuten bis wenigen Stunden herauszufinden, ob es sich überhaupt lohnt, ein Buch einer analytischen Lektüre (Stufe drei) zu unterziehen. Der erste Schritt des inspizierenden Lesens besteht darin, sich die Titelseite, das Inhaltsverzeichnis, den Umschlagtext und den Index anzuschauen, um so die Struktur des Buches kennenzulernen, und zu erfahren, wie das Buch eingeordnet werden kann. Der Leser sollte im Index nach Begriffen suchen, die ihn im Zusammenhang mit dem Thema des Buches interessieren und die jeweils referenzierten Passagen überfliegen. Oft ist es lohnenswert, die letzten zwei bis drei Seiten des Hauptteils zu überfliegen, wo viele Autoren den Inhalt des Buches rekapitulieren. Auch Vorwort und Einleitung geben oftmals Auskunft über den Aufbau des Buches.

Hat der Leser die Struktur eines Buches erfasst, soll er es im zweiten Schritt oberflächlich durchlesen. Da er bei einem anspruchsvollen Buch ohnehin bei der ersten Lektüre nicht alles verstehen wird, soll er sich zunächst auf das konzentrieren, was er bereits versteht. Ansonsten verzettelt er sich, wodurch seine Motivation schwindet. Die Lücken um das Verstandene herum werden dann beim analytischen Lesen (Stufe 3) geschlossen.

Bücher sollten nicht so schnell als möglich (Speed-Reading), sondern in einem ihnen angemessenen Tempo gelesen werden.

\subsection{Was einen anspruchsvollen Leser ausmacht}

Der anspruchsvolle Leser stellt dem Buch Fragen – und versucht sich diese mittels seiner Lektüre selber zu beantworten. Diese Fragen sind:

\begin{enumerate}
    \item Worum geht es in einem Buch als Ganzes?
    \item Worum geht es in einem Buch im Detail?
    \item Ist das Buch – im Ganzen, im Detail – wahr?
    \item Was soll das Ganze?
\end{enumerate}

Es genügt nicht, die Ansichten des Autors zu kennen. Der anspruchsvolle Leser muss auch eigene Ansichten zum Thema des Buches erarbeiten. Das Buch mit einem Stift zu lesen – Wörter, Sätze und Passagen zu markieren, Notizen zu machen – hilft dabei, aktiv zu lesen. Ein Buch gehört einem erst dann, wenn man sich nicht nur dessen physischen Besitz, sondern auch dessen Inhalt angeeignet hat. Notizen können struktureller Natur (inspizierendes Lesen), konzeptueller Natur (analytisches Lesen) oder dialektischer Natur (syntopisches Lesen) sein.

\newpage
\section*{Teil II: Die  dritte Stufe des Lesens: Analytisches Lesen}
\addcontentsline{toc}{section}{Teil II: Die  dritte Stufe des Lesens: Analytisches Lesen}

\subsection{Zur Einordnung eines Buches}

Der Leser sollte sich immer dessen bewusst sein, was für ein Buch er vor sich hat. Auf erster Stufe unterscheidet man zwischen imaginativer Literatur und Sachbuch. Sachbücher lassen sich in theoretische und praktische Bücher einteilen. Das Ziel eines theoretischen Buches ist es zu vermitteln, was und wie etwas ist – also Verständnis zu befördern: es geht ums Denken. Praktische Bücher beschreiben, wie etwas gemacht werden soll und kann: es geht ums Handeln. Ein Paradebeispiel zur Unterscheidung in theoretische und praktische Werke sind Kants erste beiden Kritiken: \textit{Die Kritik der reinen Vernunft} (Metaphysik) ist ein theoretisches Buch; \textit{Die Kritik der praktischen Vernunft} (Moral) ein praktisches Buch.

Theoretische Bücher lassen sich in folgende drei Kategorien einteilen: 1) historische Bücher, in denen es um zurückliegende Ereignisse an bestimmten Orten zu bestimmten Zeiten geht; 2) philosophische Bücher, deren Betrachtungen innerhalb der Alltagserfahrung des Menschen angesiedelt ist; und 3) (natur)wissenschaftliche Bücher, bei denen es um bestimmte aber allgemeine Situationen geht, die sich der Alltagserfahrung des Menschen entziehen.

\subsection{Ein Buch «röntgen»}

Die Frage, worum es in einem Buch gehe, kann umformuliert werden in: «Welche Absichten verfolgt der Autor?» Wie jedes Kunstwerk hat auch jedes Buch eine Einheit. Wer nach der Lektüre eines Buches in wenigen Worten oder Sätzen sagen kann, worum es darin geht, der hat diese Einheit erfasst. Weiter kann der Leser sich einen Überblick über die Struktur (das «Skelett») eines Buches verschaffen, indem er die darin behandelten Themen und Unterthemen auflistet und beschreibt. Die Struktur dieser Auflistung kann von derjenigen des Inhaltsverzeichnisses abweichen. Der Leser soll herausfinden, welche Probleme der Autor in seinem Buch behandelt bzw. welche Fragen er darin zu beantworten versucht. Der Autor soll nicht etwa einer Psychotherapie unterzogen werden; der Leser muss sich innerhalb des Abgedruckten orientieren.

\subsection{Sich mit dem Autor auf Begriffe einigen}

Der Leser muss sich mit dem Autor auf Begriffe «einigen», d.h. zu verstehen versuchen, welche Bedeutung(en) der Autor bestimmten Wörtern beimisst. Ein Wort kann für mehrere Begriffe stehen; ein Begriff kann durch mehrere Worte bezeichnet sein: zwischen Wort und Begriff liegt eine m-zu-n-Beziehung vor. Begriffe sind Wörter, die ein Autor mit Bedeutung(en) «auflädt». Es gehört zur Beherrschung der Kunst des Lesens, diese Begriffe und ihre besondere Bedeutung zu ermitteln, die sie für den Autor haben. Solche Wörter sind teils explizit mit typografischen Mitteln oder als Definitionen hervorgehoben – oder der Autor grenzt seinen Gebrauch eines Begriffs gegen dessen Verwendungen seiner Kollegen ab. Autoren verwenden auch Fachvokabular oder gar ein Privatvokabular (etwa der Seinsbegriff im Existenzialiusmus oder der Verstehensbegriff in der Hermeneutik). Wörter werden unter dem Begriff Vokabular, Begriffe unter dem Begriff Terminologie subsumiert. Ein Begriff kann mehr oder weniger breit gefasst sein und durch Synonyme bezeichnet werden.

Um die Bedeutung eines unbekannten Begriffes zu erschliessen, betrachtet der Leser die ihm bereits bekannten Begriffe wie zusammengesetzte Puzzleteile, die zwar nur ein unvollständiges Bild ergeben, das ihm aber dazu genügt, das fehlende Puzzleteil zu erraten: den unbekannten Begriff.

\subsection{Die Aussage des Autors erfassen}

Nähert sich der Leser beim Erkennen der strukturellen Organisation eines Buches von oben nach unten – also von Kapitel zu Abschnitt, von Abschnitt zu Absatz, von Absatz zu Satz und schliesslich von Satz zu Wort –, geht er bei der Interpretation in entgegengesetzter Richtung vor, also von unten nach oben: von Begriff zu These, von These zu Argument. Aus den Wörtern schöpft der Leser Begriffe, aus Sätzen Thesen und aus den Beziehungen der Sätze untereinander Argumente. Thesen sind in Schlüsselsätzen zu finden, die teils typografisch hervorgehoben sind. Ist dies nicht der Fall, muss sich der Leser selbst zu helfen wissen: wichtige Sätze sind solche, die sein Verständnis erweitern können, und ihm somit nicht nach dem ersten Durchlesen völlig verständlich sind. Sie enthalten oftmals Schlüsselbegriffe, die der Leser bereits erarbeitet hat. In der Regel sind wichtige Sätze Prämissen oder Folgerungen. Um zu überprüfen, ob der Leser eine These oder eine Argumentation verstanden hat, soll er diese in eigene Worte zu fassen versuchen. Bei der Lektüre eines fremdsprachigen Buches ist auch die freie Übersetzung eines solchen Schlüsselsatzes in die eigene Sprache eine gute Verständnisprüfung. Wer eine These verstanden hat, muss auch ein Beispiel dazu machen können.

\subsection{Ein Buch fair kritisieren}

Um ein Buch kritisieren – d.h. bewerten! – zu können, muss der Leser es zuerst verstanden haben. Es ist wichtig, das Argument eines Autors zu Ende zu verfolgen, bevor eine Wertung darüber abgegeben wird. Vor und während der Lektüre muss der Leser darauf vorbereitet sein, dass er mit dem Autor einverstanden oder nicht einverstanden sein wird. Oder er muss mit der Bewertung zuwarten, da er noch nicht über das dazu notwendige Wissen und Verständnis verfügt. Zwischen persönlicher Meinung und begründeter Ansicht besteht ein gewaltiger Unterschied. Meinungsverschiedenheiten, die von rationalen Argumenten gestützt werden, können grundsätzlich überwunden werden.

\subsection{Einem Autor zustimmen oder widersprechen}

Für eine faire Bewertung eines Buches muss es nicht «nur» verstanden worden sein, es erfordert auch weitere zwingende Vorbedingungen: 1) Der Leser muss sich dessen bewusst sein, dass Menschen von Gedanken und von Gefühlen geleitet werden, letztere für die Bewertung jedoch nicht ausschlaggebend sein dürfen. 2) Der Leser muss die Prämissen des Autors zu akzeptieren versuchen, um dessen Argumentation nachvollziehen und bewerten zu können. 3) Der Leser muss ein Buch wohlwollend lesen, d.h. versuchen, den Gesichtspunkt des Autors einzunehmen.

Es gibt vier Gründe, mit einem Autor nicht zufrieden zu sein: er ist in seinem Buch 1) uninformiert, 2) fehlinformiert, 3) unlogisch oder 4) unvollständig. Die ersten beiden Punkte hängen insofern zusammen, als dass der Autor nicht über die Wahrheit verfügt. Man beachte, ob das Wissen zum Zeitpunkt der Niederschrift schon vorhanden war! (Es wäre unfair, Darwin für die Unkenntnis der genetischen Vererbungslehre zu rügen, die erst viel später von Mendel entdeckt wurde.) Unlogisch ist die Argumentation, wenn aus (wahren oder falschen) Prämissen die falschen Schlüsse gezogen werden, d.h. der Autor die Regeln der formalen Logik nicht einhält. (Beispiel: Alle Affen sind Säugetiere, ergo sind alle Säugetiere Affen.) Eine unlogische Argumentation liegt auch dann vor, wenn diese inkonsistent ist, d.h. Widersprüche enthält. Das Werk ist unvollständig, wenn der Autor nicht das hält, was er eingangs versprochen hat: er beantwortet die von ihm gestellte Frage nicht (zureichend) – oder er hat nicht alles aus seinem Material herausgeholt. Bei den Punkten eins bis drei widerspricht der Leser zu einem gewissen Grad dem Autor, bei Punkt vier ist er zwar mit ihm einverstanden, jedoch nicht zufrieden.

\subsection{Hilfsmittel zum Lesen}

Analytische Lektüre ist intrinsische Lektüre. Der Leser geht dabei nicht über das zu lesende Buch hinaus. Bedient er sich weiteren Wissens, das nicht im Buch vorhanden ist, spricht man von extrinsischer Lektüre. Extrinsische Hilfsmittel sind:

\begin{description}
\item[Erfahrungen] allgemeiner oder spezieller Natur. Erstere sind wichtig für die Lektüre von Philosophie und imaginativer Literatur und umfassen die Alltagserfahrungen des Menschen; zweitere umfassen aus Forschung gewonnene (empirische) Erkenntnisse, die nicht jedem zur Verfügung stehen. Solche Erfahrungen sind für die Lektüre (natur)wissenschaftlicher Werke notwendig. Für historische Werke braucht der Leser aus beiden Erfahrungsschätzen zu schöpfen.
\item[Bücher] können nicht isoliert betrachtet werden, denn die meisten Autoren sind auch Leser anderer Bücher. Wichtig ist die chronologische Reihenfolge des Erscheinens einzelner, teils aufeinander aufbauender Bücher zu einem Wissensgebiet.
\item[Kommentare und Lektürehilfen] sollten erst nach der Lektüre des Buches zuhilfe gezogen werden, da der Leser sonst in eine bestimmte Interpretationsrichtung gedrängt werden und dann das Buch nicht mehr unvoreingenommen lesen könnte. Kommentare sind zudem – wie jedes Buch – möglicherweise fehlerhaft und unvollständig. Inhaltsangaben können dem Leser dabei helfen, den Inhalt eines Buches aufzufrischen. Abstracts dienen als Entscheidungshilfe, ob ein (zusätzliches) Buch überhaupt gelesen werden soll.
\item[Referenzen] verlangen vom Leser a) dass er weiss, was er erfahren will – welche Wissenslücke will er durch die Konsultation stopfen? b) Kenntnis über die verschiedenen Arten von Referenzwerken im Allgemeinen und c) Kenntnis über das zu konsultierende Referenzwerk im Besonderen. Der Leser muss auch wissen, welche Arten von Fragen überhaupt mit Referenzwerken beantwortet werden können.
\item[Wörterbücher] klären über Rechtschreibung, Aussprache, Bedeutungen, Gebrauch und Herkunft von Wörtern auf.
\item[Enzyklopädien] liefern dem Leser «harte» Fakten über Sachverhalte, über die sich die Menschheit weitgehendst einig ist.Über Fakten sollte nicht gestritten werden – man sollte sie besser nachschlagen.
\end{description}

\newpage
\section*{Teil III: Ansätze für verschiedene Arten von Büchern}
\addcontentsline{toc}{section}{Teil III: Ansätze für verschiedene Arten von Büchern}

\subsection{Wie man praktische Bücher liest}

Je weiter Regeln gefasst werden, desto weiter sind sie von ihrem Anwendungsgegenstand entfernt. Unterschiedliche Arten von Büchern erfordern präzisierte Regeln. Gerade imaginative Literatur lässt sich mit den bisher geschilderten Regeln kaum fassen.

Auch für das Lesen praktischer Bücher gelten besondere Regeln. Im Gegensatz zu theoretischen Büchern, die ein theoretisches Problem formulieren und zu lösen versuchen, lösen praktische Bücher selber keine Probleme, sondern bieten dem Leser Handlungsanweisungen, damit er damit Probleme lösen kann. Es gibt zwei Arten von praktischen Büchern, wobei der Übergang fliessend ist: 1) Bücher, die Regeln und Handlungsweisen darlegen, und 2) Bücher, die sich mit den Prinzipien befassen, aus denen Regeln abgeleitet werden. Die Thesen praktischer Bücher sind oft die Regeln selbst, die man anhand ihres imperativen Aussagemodus erkennt. Die Argumentation umfasst dann die Begründung dieser Regeln.

Für die Bewertung praktischer Bücher gibt es zwei Kriterien: 1) die Handlungsanweisungen funktionieren, und 2) sie führen zu einem erwünschten Ziel. Das Ziel fungiert hier als Prämisse, mit der ein Leser möglicherweise nicht einverstanden ist. Schliesslich kann der Leser andere Ziele verfolgen, d.h. die vom Autor beschriebenen Absichten nicht gutheissen. Bei der Lektüre von praktischen Büchern muss sich der Leser zwei Fragen stellen: 1) Welche Ziele gibt der Autor vor? 2) Welche Mittel bietet er zu deren Erreichung an? In einem praktischen Buch versucht der Autor immer den Leser zu überzeugen. Dazu bedient er sich auch Rhetorik, die auf Gefühle, und nicht auf den Verstand abzielt. Solche Rhetorik ist nur dann gefährlich, wenn der Leser sie nicht als solche erkennt. Es ist von Vorteil, die Hintergründe und Lebensumstände des Autors zu kennen.

Die Frage, welche Probleme der Autor zu lösen versucht (Regel 4 des analytischen Lesens), kann umformuliert werden in: «Wozu möchte mich der Autor verleiten, welche Ziele verfolgt er mit mir?» Die Frage, ob der Autor sein Problem gelöst hat (Regel 8 des analytischen Lesens), lautet dann folglich: «Welche Mittel gibt mir der Autor an die Hand, und sind diese hinreichend?» Die Frage, was denn das alles soll (Frage 4 des aktiven Lesens), wird dahingehend beantwortet, dass der Leser seine Ansichten über einen Sachverhalt ändert und seine Handlungsweise entsprechend anpasst.

\subsection{Wie man imaginative Literatur liest}

Das Lesen imaginativer Literatur erfordert eine Anpassung der bisher geschilderten Regeln. Imaginative Literatur vermittelt nicht Wissen, sondern Erlebnisse. Sie erfordert nicht Urteilsvermögen, sondern Vorstellungskraft. Damit sich die Vorstellung des Lesers entfalten kann, darf er der Wirkung, die das Werk auf ihn ausübt, nicht zu widerstehen versuchen. Imaginative Literatur lebt von sprachlichen Ambiguitäten: Zwischen den Zeilen steht mindestens so viel wie in den Zeilen. Die Aussage eines imaginativen Werkes besteht nicht im abgedruckten Text allein. Imaginative Literatur erfordert einen angepassten Wahrheitsbegriff: Das Geschilderte muss nicht wahr sein im Sinne des faktisch Überprüfbaren, es muss plausibel sein in der Welt, die der Autor errichtet.

Ein imaginatives Werk lässt sich einordnen (Roman, Theaterstück usw.) und hat eine Einheit: den Plot, der sich in wenigen Sätzen als Inhaltsangabe beschreiben lässt. Er umfasst das, was in den Zeilen, nicht zwischen den Zeilen steht. Die Handlung imaginativer Werke ist nicht logisch-strukturell angeordnet, sondern chronologisch in eine bestimmte (nicht zwingend lineare) Ordnung gebracht. Statt Begriffe ermittelt der Leser Episoden, Vorfälle, Charaktere, Gedanken, Reden, Gefühle und Handlungen. Statt Thesen aufzuspüren muss sich der Leser in die geschilderte Welt hineinversetzen. Statt einer Argumentation entwickelt sich eine Handlung.

Auch der Leser eines imaginativen Werkes darf dieses erst dann kritisieren, wenn er es verstanden hat. Dazu muss er zumindest versuchen, das Geschilderte nachzuempfinden. Wie er in einem Sachbuch die Prämissen akzeptieren muss, um die Argumentation nachzuvollziehen, muss er in einem imaginativen Werk die Welt hinnehmen, wie sie ihm der Autor vorlegt, damit er die Handlung nachvollziehen kann. Bei imaginativer Literatur lautet das Urteil nicht auf «einverstanden»/«nicht einverstanden», sondern auf «gefällt mir»/«gefällt mir nicht», wobei Zustimmung und Ablehnung begründet werden müssen.

\subsection{Vorschläge für das Lesen von Prosa, Theaterstücken und Gedichten}

Die vierte Frage des aktiven Lesens wird für fiktionale Literatur anders beantwortet als für Sachbücher. Sachbücher verleiten den Leser oftmals zum Handlen, auch wenn diese Handlung bei theoretischen Büchern im blossen Verstehen besteht. Fiktionale Literatur enthält meist keine solche Handlungsabsicht, obwohl sie das Handeln oder das Weltbild des Leser beeinflussen kann. Ihr Ziel ist meist die Schönheit als Selbstzweck.

Ein Prosawerk sollte zunächst einmal möglichst an einem Stück oder in einer möglichst kurzen Zeitspanne durchgelesen werden. Auf diese Weise verpasst der Leser den Anschluss nicht und kann am besten in die geschilderte Welt eintauchen. Nach dieser ersten oberflächlichen – naiven, inspizierenden – Lektüre kennt der Leser die Einheit des Werkes und weiss, was darin wichtig ist und welche Figuren wichtig sind. Die Details erschliessen sich dann in der zweiten, gründlichen – analytischen – Lektüre.

Die meisten Theaterstücke enthalten weniger detaillierte Beschreibungen als Prosawerke. Der Leser muss sich mit meist spärlichen Regieanweisungen begnügen und benötigt darum eine besonders starke Vorstellungskraft. Ein Theaterstück wird erst dann zum vollständigen Kunstwerk, wenn es auf der Bühne aufgeführt wird. Diese Lücke kann der Leser für sich schliessen, indem er Regisseur spielt, und sich überlegt, wie er es inszenieren würde.

Ein Gedicht sollte zunächst einmal durchgelesen werden, ohne irgendwo stehen zu bleiben. Viele Leser machen den Fehler, bei der ersten Lektüre alles verstehen zu wollen und sind dann bei den ersten Schwierigkeiten bereits frustriert. Beim ersten Durchlesen kann aber bei weitem nicht alles, immerhin aber die Einheit des Gedichts erfasst werden. Das zweite Durchlesen sollte laut erfolgen. So kann der Leser nicht über einzelne (ihm unverständliche) Wörter hinweglesen. Das Ohr entlarvt fehlerhafte Betonungen, die dem Auge nicht auffallen. Laut vorgelesen offenbart sich die syntaktische Struktur eines Gedichts deutlicher als beim stillen Durchlesen. Ist der Leser bei Sachbüchern mit grammatikalischen und logischen Fragen konfrontiert, sind es bei Gedichten rhetorische und syntaktische Fragen. Er sucht nicht nach Begriffen, sondern nach Schlüsselwörtern, die durch Reim, Rhytmus und Wiederholung hervorgehoben sein können. Viele Gedichte enthalten einen Konflikt, den es zwischen den Zeilen zu entdecken gilt. Bei Gedichten sollte sich der Leser nicht uneingeschränkt auf Sekundärliteratur stützen, sondern sich das Gedicht durch abermalige Lektüre zu ergründen versuchen. Das Lesen und Verstehen von Gedichten ist ein Lebensprojekt.

\subsection{Wie man Geschichte liest}

Geschichte ist der narrative Zugang zu einer Epoche, zu einem Ereignis oder zu einer Reihe von Ereignissen in der Vergangenheit, der in einer mehr oder weniger formalen Weise präsentiert wird. Historiker haben unterschiedliche Theorien und Weltanschauungen. Für die einen lassen sich Ereignisse in übergeordnete Muster und Entwicklungen einordnen. Für die anderen ist Geschichte ein einziges Chaos, aus dem niemand bestimmen kann, warum ein bestimmes Ereignis eingetreten (oder ausgeblieben) ist. Es ist wichtig, die theoretische Position des Autors zu kennen! Um sich ein qualifiziertes Bild über ein historisches Thema verschaffen zu können, müssen mehrere Werke von unterschiedlich disponierten Autoren gelesen werden. Jede Erzählung – auch Geschichte – ist aus einer bestimmten Perspektive geschrieben, die der Leser zum Verständnis unbedingt kennen muss. Geschichte führt uns nicht nur vor Augen, was sich wann und wo – und warum – ereignet hat, sondern auch, wie sich Menschen in bestimmten Situationen verhalten – auch in der Gegenwart.

Bei einem Buch über Geschichte muss sich der Leser vergewissern, wie der Autor sein Thema eingrenzt, welche Aspekte er behandelt und wie er sie gewichtet. Die Kritik kann an zwei Stellen ansetzen: 1) Der Autor interpretiert die Fakten mehr oder weniger glaubwürdig, und 2) er ist mehr oder weniger gut informiert. Die vierte Frage des aktiven Lesens – was das Ganze soll – ist damit zu beantworten, dass Geschichte und ihre Überlieferung einen gewaltigen Einfluss auf politisches Handeln ausübt. Darum ist es wichtig, Geschichte besonders gut zu lesen, damit man aus den Fehlern der Vergangenheit lernt, statt sie zu wiederholen.

Biografien sind Geschichten über reale Persönlichkeiten – also wie jede Geschichtsschreibung Realität und Fiktion zugleich. Eine definitive Biografie wird meist von einem Gelehrten viele Jahre nach dem Tod der Persönlichkeit verfasst, oftmals unter Prüfung deren Korrespondenz und Befragung von Zeitgenossen. Autorisierte Biografien entstehen zu Lebzeiten, sind somit nicht vollständig und in der Regel voreingenommen, möchte sich doch darin eine Persönlichkeit ins beste Licht rücken. Manche Biografien haben auch eine pädagogische Stossrichtung und verfolgen moralische Absichten, wie z.B. Plutarchs \textit{Grosse Griechen und Römer}. Autobiografien sind naturgemäss nie definitiv, enthalten selten die ganze Wahrheit, aber immer einen Teil davon – auch wenn dieser zwischen den Zeilen aufgespürt werden muss.

Bei Büchern (und Artikeln) über aktuelle Ereignisse muss sich der Leser vor Augen führen, wer der Autor ist. Er muss sich folgende Fragen stellen: 1) Was möchte der Autor nachweisen? Der Klappentext oder das Vorwort geben meist darüber Auskunft. 2) Wen möchte der Autor überzeugen? Gehört der Leser zu seiner «Zielgruppe»? 3) Welches Wissen – oder welche Vorurteile und Meinungen, welches Weltbild setzt der Autor voraus? Kann der Leser seine Prämissen überhaupt akzeptieren, kann er das Buch überhaupt gewinnbringend lesen? 4) Bedient sich der Autor einer besonderen Sprache? Macht er Gebrauch von Kampfbegriffen? 5) Weiss der Autor überhaupt, worüber er schreibt? Oder verfolgt er insgeheim andere Absichten, als er es vorgibt zu tun? Oder zusammengefasst: Was möchte der Autor erreichen? Diese Frage sollte sich auch ein Leser von Zusammenfassungen und «Digests» stellen.

\subsection{Wie man Naturwissenschaft und Mathematik liest}

Bis zum Ende des 19. Jahrhunderts wurden die meisten naturwissenschaftlichen Bücher für ein Laienpublikum geschrieben. Heute richten Naturwissenschaftler ihre Schriften zumeist an ein wissenschaftliches Publikum. Der Laie muss sich aber nicht mit populärwissenschaftlichen Büchern begnügen, sondern kann auch die Klassiker der Naturwissenschaft gewinnbringend lesen, wenn er willens ist, einiges an Zeit und Anstrengung auf seine Lektüre zu verwenden. Dazu muss sich der Leser besonders gut vor Augen halten, welches Problem der Autor zu lösen versucht. Die Lektüre der naturwissenschaftlichen Klassiker bringt dem Leser keine nennenswerten Erkenntnisse auf dem jeweiligen Fachgebiet, dafür erfährt er einiges über die Philosophie und die Geschichte der Wissenschaft.

Gute wissenschaftliche Untersuchungen unterscheiden klar zwischen Annahmen und Erkenntnissen. Der Autor legt seine Prämissen deutlich dar. In den Naturwissenschaften ist die induktive Methode sehr verbreitet, d.h. aus einer Beobachtung (z.B. aus einem Experiment oder einer Feldstudie) wird ein Gesetz abgeleitet. Die Schwierigkeiten im Verständnis liegen meist darin, dass der Leser die beschriebenen Schritte nicht nachvollziehen kann, da ihm die dazu nötigen Einrichtungen fehlen.

In mathematischen Büchern geht es vorallem um das Beweisen von Thesen. Eine These ist eine wenn-dann-Beziehung, die aussagt, dass wenn eine Hypothese wahr ist, folglich auch eine Schlussfolgerung zutreffen muss. Der Wahrheitsgehalt der Hypothese ist dabei nicht von Belang. Bei mathematischen Büchern ist es besonders wichtig, sie von Anfang bis Ende zu lesen und sich Notizen zu machen, wozu separate Blätter empfehlenswert sind. Die mathematischen Teile von naturwissenschaftlichen Büchern können teilweise übersprungen werden, ohne dass der Leser Einbussen im Verständnis des Hauptarguments in Kauf nehmen muss. Mathematik dient in naturwissenschaftlichen Büchern häufig «nur» zum überprüfen, beweisen und illustrieren von Teilargumenten.

Populärwissenschaftliche Bücher sind meist einfacher zu lesen als rein wissenschaftliche. Einerseits werden detaillierte Beschreibungen von Experimentanordnungen ausgespart und nur die Ergebnisse dieser Experimente geschildert. Andererseits enthalten sie selten anspruchsvolle Mathematik. Die Kehrseite dieser Vereinfachung ist, dass dem Leser die Informationen fehlen, um die Argumentation nachvollziehen zu können. Darum ist er bei populärwissenschaftlichen Büchern vollständig auf den Autor angewiesen und muss diesem blind vertrauen.

\subsection{Wie man Philosophie liest}

In philosophischen Büchern werden scheinbar einfache Fragen behandelt, die sich als schwer bis unmöglich zu beantworten erweisen. Solche Fragen lassen sich in zwei Gruppen aufteilen: 1) Fragen der theoretischen oder spekulativen Philosophie über das Sein und das Werden – über das, was ist; und 2) Fragen der praktischen oder normativen Philosophie über Gut und Böse – über das, was sein soll. Mit Fragen der ersten Gruppe befassen sich die Metaphysik (Fragen über das Sein und die Existenz); die Naturphilosophie (Fragen über das Werden); und die Epistemologie (Fragen über das menschliche Wissen), die Theorie des Wissens. Mit Fragen der zweiten Gruppe befassen sich die Ethik (Fragen über Gut und Böse, über das gute Leben des Individuums); und die politische Philosophie (Fragen über die gute Gesellschaft, das Zusammenleben der Individuen miteinander).

Solche Fragen bezeichnet man als Fragen erster Ordnung. Fragen zweiter Ordnung befassen sich mit unserem Wissen, unseren Denkinhalten und unseren sprachlichen Äusserungen über Fragen erster Ordnung. Werke, in denen Fragen erster Ordnung behandelt werden, sind – die nötige Leseanstrengung vorausgesetzt – dem Laien zugänglich. Dazu gehören die klassischen Werke der Philosophie und die meisten bis ca. 1930 erschienenen philosophischen Werke. Die modernere Philosophie befasst sich eher mit Fragen zweiter Ordnung. Sie wurden und werden von Experten für Experten geschrieben.

Wahre philosophische Fragen können nur durch reines Nachdenken beantwortet werden. Der Leser muss unterscheiden, welche in einem philosophischen Werk untersuchten Fragen tatsächlich philosophischer Natur sind. Bei den alten Griechen gehörten etwa Fragen über die Himmelskörper noch in den Kompetenzbereich der Philosophie und wurden spekulativ behandelt. Heute sind diese Fragen Gegenstand der Naturwissenschaft, sprich der Astronomie.

Der Leser sollte sich den Stil eines philosophischen Werkes vergegenwärtigen: Platon prägte den Dialogstil und gilt darin bis heute als unübertroffen. In Dialogen wird eine Frage in einem (zuweilen informellen) Gespräch erörtert. Abhandlungen, z.B. diejenigen Kants und Aristoteles', untersuchen ein eingangs formuliertes Problem in einer systematischen Untersuchung, die geradliniges Durchlesen erfordert. Bei der von Thomas' von Aquin in seiner \textit{Summa Theologica} angewanden Form der Gegenüberstellung von Argumenten wird zuerst eine Frage gestellt, dann falsch beantwortet. Es folgen Argumente für diese falsche Antwort, die anschliessend widerlegt werden. Zum Schluss folgt die richtige Antwort auf die eingangs gestellte Frage. Descartes pflegte einen sehr systematischen, Spinoza einen geradezu mathematischen Stil mit Thesen und Beweisführungen. Nietzsche pflegte einen teils aphoristischen Stil, etwa in \textit{Also sprach Zarathustra}, der aufgrund seiner poetischen Form Lesen zwischen den Zeilen erfordert – und darum aus rein philosophischer Sicht als unbefriedigend gilt.

Bei philosophischen Büchern ist es besonders wichtig, die Fragen, die sich ein Autor stellt, genau zu kennen. Diese sind nicht zwingend explizit gestellt, sondern können im Text verborgen liegen. Das gleiche gilt für die Annahmen und Grundprinzipien, von denen ein Autor ausgeht. Nicht jeder Philosoph ist sein ganzes Werk hindurch konsistent was seine Grundprinzipien angeht! Stösst der Leser auf eine solche Inkonsistenz, muss er herausfinden, welche Prinzipien der Autor im jeweiligen Fall gelten lässt. Der Leser muss acht geben auf Wörter, die in einem ungewöhnlichen, nicht-alltäglichen Sinn gebraucht werden und ihre Bedeutungen im jeweiligen Kontext ermitteln. Besonders wichtig ist es für den Leser philosophischer Bücher, sich mit dem Autor auf Begriffe zu «einigen» (Regel 5 des analytischen Lesens). Um der Argumentation folgen zu können muss der Leser dazu bereit sein, Prämissen zeitweilig zu akzeptieren – selbst wenn ihm diese als falsch erscheinen. Dies kann sich als lohnende mentale Übung erweisen. Annahmen, die nicht explizit im Text belegt sind, entspringen für gewöhnlich aus der alltäglichen Erfahrung. Um solche Annahmen überprüfen zu können, muss sich der Leser nur auf seinen eigenen Erfahrungsschatz berufen – Kommentare und Erläuterungen braucht er dazu nicht. Bei philosophischen Büchern ist der Leser auf seinen eigenen Verstand angewiesen wie bei kaum einer anderen Textsorte.

Philosophische Bücher stehen nicht für sich allein, da sich verschiedene Philosophen mit ähnlichen Fragestellungen beschäftigt haben. Der Leser tut gut daran, mehrere Bücher aus einem Bereich zu lesen, bevor er sein Urteil fällt. Widersprüche zwischen einzelnen Werken deuten nicht zwingend auf Fehler in denselben hin, sondern vielmehr auf übergeordnete – unlösbare? – Probleme. Philosophische Fragen muss jeder für sich selber zu beantworten versuchen. Das Übernehmen von «Expertenmeinungen» bedeutete hier den philosophischen Fragen auszuweichen.

Bei theologischen Büchern unterscheidet man zwischen Naturtheologie, die zur Methaphysik (und somit zur Philosophie) gehört und eine Gottheit an den Anfang jeder Kausalkette setzt; und dogmatischer Theologie, die von bestimmten, von einer religiösen Autorität vorgegeben Glaubenssätzen (den Dogmen oder Dogmata) ausgeht. Gehört der Leser eines dogmatischen Buches nicht der jeweiligen Glaubensrichtung an, kann er ein solches Werk dennoch mit Erkenntnisgewinn lesen, schliesslich sind Dogmen nichts weiter als Prämissen, die man für das Verfolgen einer Argumentation (zeitweilig) akzeptieren muss. Nur die Prämissen sind dogmatisch – die Argumentation darf es nicht sein!

Kanonische Bücher setzen eine bestimmte – orthodoxe – Lesart voraus, um für «wahr» befunden zu werden. Für die Christen ist \textit{Die Bibel}, für Moslems \textit{Der Koran}, für Marxisten \textit{Das Kapital}, für Maoisten \textit{Das Rote Buch} ein kanonisches Buch. Der gläubige Leser ist dazu gezwungen, ein kanonisches Buch «richtig» zu verstehen – und wenn er dazu der Anleitung eines Dritten bedarf.

\subsection{Wie man Sozialwissenschaften liest}

Der Begriff Sozialwissenschaften umfasst im Kern Anthropologie, Ökonomie, Politik und Soziologie, wobei häufig auch Recht, Bildung, öffentliche Verwaltung, Sozialarbeit, Themen der Betriebswirtschaft und der Psychologie hinzugezählt werden. Das Lesen sozialwissenschaftlicher Bücher erscheint zunächst als einfach, da in ihnen Daten aus uns bekannten Erfahrungsbereichen verwendet werden, der narrative Stil ihrer Darlegung zugänglich ist und die meisten Leser mit ihrem Jargon vertraut sind. Diskussionen über soziale Themen sind etwas Alltägliches. Fast jeder hat schon über soziale Fragen gelesen, diskutiert und sich eine Meinung dazu gebildet. Doch hat diese scheinbare Einfachheit eine Kehrseite: Gerade weil (fast) jeder eine Meinung zu sozialen Fragestellungen hat, fällt es besonders schwer, unvoreingenommen an ein sozialwissenschaftliches Buch heranzugehen und anderslautende Prämissen zum Zweck des Nachvollziehens der Argumentation zeitweilig zu akzeptieren. Und selbst wenn der Leser die Argumentation gutheisst, ist er aufgrund seiner Vorurteile dazu geneigt, sie dennoch abzulehnen. Zwar erscheint dem Leser das sozialwissenschaftliche Jargon zunächst als vertraut, doch erweisen sich die Begriffe oftmals als schwamming und sind selten genauer definiert, wie es bei den Naturwissenschaften der Fall ist. (Was bedeutet eigentlich «Gesellschaft»?) Der Stil sozialwissenschaftlicher Untersuchungen ist zuweilen eine Mixtur aus Naturwissenschaft, Geschichte, Philosophie – und einem guten Stück Fiktion, was die Einordnung sozialwissenschaftlicher Literatur erschwert. Oft fehlt es an Standardwerken zu bestimmten Themengebieten, was die Lektüre mehrerer Bücher zum gleichen Thema erfordert – und somit Gegenstand des vierten Teils ist.

\newpage
\section*{Teil IV: Die höchsten Ziele des Lesens}
\addcontentsline{toc}{section}{Teil IV: Die höchsten Ziele des Lesens}

\subsection{Die vierte Stufe des Lesens: Syntopisches Lesen}

Beim syntopischen Lesen geht es um das Lesen mehrerer Bücher zum selben Thema. Die Voraussetzungen dafür sind: 1) zu wissen, dass mehrere Bücher zu einem Thema gelesen werden sollen; und 2) zu wissen, um welche Bücher es sich dabei handelt. Die zweite Voraussetzungen ist ungleich schwieriger zu erfüllen als die erste, da die Menge der Bücher zu manchen Themen kaum überschaubar ist. Der Leser muss aber irgendwo beginnen und kann sein Thema im Verlauf der Lektüre besser eingrenzen. In diesem Prozess wird er auch Bücher lesen, die am Ende nicht mehr zu dem nunmehr eingegrenzten Themengebiet gehören. Dies ist das Paradoxon des syntopischen Lesens: Die Identifikation des Themas geht der Lektüre voraus und folgt ihr.

Hat sich der Leser eine Bibliografie mit mutmasslich für sein Thema relevanten Büchern zusammengestellt, soll er diese zunächst inspizierend lesen. Dadurch kann er einerseits sein Thema weiter eingrenzen und andererseits besser einschätzen, ob die Bücher auf seiner Liste für ihn einen Erkenntnisgewinn bergen oder nicht – und somit seine Bibliografie verkleinern.

Das syntopische Lesen erfolgt in fünf Schritten:

\begin{description}
    \item[Auffinden der relevanten Passagen] Beim syntopischen Lesen geht es zunächst um ein Thema und erst dann um einzelne Bücher. Die Bücher müssen erneut inspiziert werden, um darin Passagen aufzuspüren, die zum Verständnis des gestellten Problems beitragen können. Verhält sich der Leser beim analytischen Lesen zu seinem Buch wie ein Jünger zu seinem Meister, muss der Leser beim syntopischen Lesen die Rolle des Meisters übernehmen.
    \item[Die Autoren zu Begriffen führen] Versucht der Leser sich beim analytischen Lesen mit dem Autor auf Begriffe zu einigen (siehe Regel 5 des analytisches Lesens), muss sich der Leser beim syntopischen Lesen angesichts der Vielzahl von Autoren – und Terminologien – seine eigenen Begriffe schaffen. Er muss die Begriffe verschiedenster Autoren mit seiner eigenen Terminologie fassbar machen und dabei zwischen den einzelnen Terminologien übersetzen können wie von einer Sprache in die andere.
    \item[Die Fragen klären] Wie sich der Leser seine eigene Terminologie zurechtlegt, braucht er auch seine eigenen Thesen zu formulieren. Dazu kann er sich Fragen ausdenken, die zum Verstehen des Themas beantwortet werden müssen, und prüfen, ob und wie die Autoren diese beantworten. Das Problem hierbei liegt darin, dass sich die Autoren nicht die gleichen Fragen gestellt haben wie der Leser. Darum muss er besonders aufmerksam sein und die Antworten auf seine Fragen auch zwischen den Zeilen suchen.
    \item[Die Streitfälle ermitteln] Macht der Leser bei verschiedenen Autoren unterschiedliche Antworten auf die gleiche Frage aus, hat er einen Streitfall ermittelt. Dies muss nicht zwingend ein Widerspruch sein, da verschiedene Autoren grundverschiedene Ansichten zu einem Themengebiet haben können. Findet der Leser mehrere Streitfälle zu einer Reihe zusammenhängender Fragen, ist der einer Kontroverse auf der Spur.
    \item[Analyse der Diskussion] Die Wahrheit über ein Problem des syntopischen Lesens kann kaum in der Antwort eines einzelnen Autors bestehen, sondern in der Diskussion selbst. Ist der Leser mit unterschiedlichen, durch überzeugende Argumente gestützten Antworten konfrontiert, liegt die Lösung seines Problems in der Kenntnis der Diskussion selber. Kann er aufzeigen, wie die einzelnen Fragen von verschiedenen Autoren beantwortet werden und warum sie so beantwortet werden, hat er das Problem verstanden. Mehr ist nicht zu tun, da die Probleme des syntopischen Lesens kaum abschliessend zu lösen sind.
\end{description}

Beim syntopischen Lesen geht es nicht darum, sich auf einen Standpunkt zu stellen und alle anderen Ansichten zu widerlegen. Das Ergebnis einer solchen Untersuchung wäre nicht ein objektiver Überblick über ein Problem, sondern bloss ein weiterer Standpunkt in der Beurteilung desselben. Der Leser soll sich nicht auf eine Seite schlagen, sondern alle Seiten untersuchen. Dazu kann es hilfreich sein, Interpretationen mit einem Zitat des jeweiligen Autors zu belegen.

\subsection{Lesen und das Wachsen des Verstandes}

Wer seine Lesefähigkeit verbessern will, kann dazu nicht irgendwelche Bücher lesen, sondern nur solche, die ihm «zu hoch» sind und zu seinem Verständnis beitragen können – und nicht nur zur Unterhaltung oder zur Information dienen. Solche Bücher müssen anspruchsvoll sein. Meistert der Leser sie, verbessert er dadurch nicht nur seine Lesefähigkeit, sondern erlangt auch Wissen und Verständnis über die Welt und sich selber – er erlangt Weisheit.

Von den millionen bis dato gedruckten Büchern vermögen nur einige tausend etwas zum Verständnis beizutragen. Darunter gibt es ein paar hundert, zu denen der Leser auch nach erstmaliger gründlicher Lektüre zurückkehrt. Von den meisten dieser Bücher wird er bei abermaliger Lektüre enttäuscht sein, weil sein Verstand in der Zwischenzeit über diese Bücher hinausgewachsen ist. Nur ganz wenige Bücher scheinen mit dem Leser zu «wachsen». Sie können auf mehreren Stufen zugänglich sein, wovon der Leser eine dieser Stufen nach der anderen erklimmt, ohne das Buch je «ausgelesen» zu haben. Dies sind die berühmten zehn Bücher, die man auf eine einsame Insel mitnehmen würde.

Im Gegensatz zum menschlichen Körper kann sich der Verstand bis ins hohe Alter weiterentwickeln. Der Verstand verhält sich wie ein Muskel: wird er beansprucht, wächst er – bleibt er ungenutzt, verkümmert er. Gutes Lesen ist nicht nur ein reiner Selbstzweck. Es behält den Verstand am Leben und trägt zu seiner Entwicklung bei.

\newpage
\section*{Die Regeln im Überblick}
\addcontentsline{toc}{section}{Die Regeln im Überblick}

\subsection*{Fragen für das aktive Lesen}

\begin{enumerate}
    \item Worum geht es in einem Buch als Ganzes?
    \item Worum geht es in einem Buch im Detail?
    \item Ist das Buch – im Ganzen, im Detail – wahr?
    \item Was soll das Ganze?
\end{enumerate}

\subsection*{Regeln für das inspizierende Lesen}

\begin{itemize}
    \item Stufe 1 des inspizierenden Lesens: Systematisches Skimming
    \begin{enumerate}
        \item Betrachte die Titelseite und, falls vorhanden, das Vorwort.
        \item Untersuche das Inhaltsverzeichnis, um dir einen Überblick über die Struktur des Buches zu verschaffen.
        \item Überprüfe den Index, um abzuschätzen, welche Themen im Buch abgehandelt und auf welche weitere Bücher verwiesen wird.
        \item Lese auch den Klappen- oder Umschlagstext. Dieser besteht nicht immer nur aus reinem Marketinggeschwätz.
        \item Wirf einen Blick auf die Kapitel, die dir als ausschlaggebend erscheinen.
        \item Blättere im Buch, lese ein paar Abschnitte bis wenige Seiten an verschiedenen Stellen. Lese die letzten Seiten des Hauptteils.
    \end{enumerate}
    \item Stufe 2 des inspizierenden Lesens: Oberflächliches Durchlesen
    \begin{enumerate}
        \item Lese das Buch einmal durch und konzentriere dich dabei auf das, was du bereits verstehst, und ignoriere, was du noch nicht verstehst.
    \end{enumerate}
\end{itemize}

\subsection*{Regeln für das analytische Lesen}

\begin{itemize}
    \item Stufe 1 des analytischen Lesens: Erfahren, worum es in einem Buch geht
    \begin{enumerate}
        \item Ordne das Buch gemäss Art (fiktive Literatur, Sachbuch) und Thema ein.
        \item Beschreibe so kurz als möglich, worum es in diesem Buch geht.
        \item Liste seine Haupt- und Unterteile in ihrer Ordnung und Beziehungen zueinander auf.
        \item Definiere, welches Problem (oder welche Probleme) der Autor in seinem Buch zu lösen versucht.
    \end{enumerate}
    \item Stufe 2 des analytischen Lesens: Interpretation des Inhalts
    \begin{enumerate}
        \item «Einige» dich mit dem Autor auf Begriffe, indem du seine Schlüsselwörter interpretierst.
        \item Erfasse die Behauptung(en) des Autors, indem du die Schlüsselsätze untersuchst.
        \item Erkenne die Argumentation des Autors, indem du sie aufspürst und/oder aus seinen Thesen konstruierst.
        \item Finde heraus, welche Probleme der Autor gelöst hat und welche nicht, und im zweiten Fall, ob sich der Autor dieses «Mangels» bewusst ist.
    \end{enumerate}
    \item Stufe 3 des analytischen Lesens: Kritik des Buches
    \begin{enumerate}
        \item Allgemeine Maximen intellektueller Etikette
        \begin{enumerate}
            \item Kritisiere (d.h. bewerte!) erst, wenn du verstanden hast.
            \item Widerspreche nicht in streitlustiger Manier.
            \item Streiche den Unterschied zwischen Wissen und persönlicher Meinung hervor, indem du rationale Gründe für deine Urteile angibst.
        \end{enumerate}
        \item Spezielle Kriterien für die Äusserung von Widerspruch
        \begin{enumerate}
            \item Zeige auf, in welchen Punkten der Autor uninformiert ist.
            \item Zeige auf, in welchen Punkten der Autor fehlinformiert ist.
            \item Zeige auf, wo der Autor unlogisch argumentiert.
            \item Zeige auf, wo das Werk unvollständig ist.
        \end{enumerate}
    \end{enumerate}
\end{itemize}

\subsection*{Stufen des syntopischen Lesens}

\begin{enumerate}
    \item Die Bücher auswählen
    \begin{enumerate}
        \item Erstelle eine Bibliografie zu deinem Thema mithilfe von Bibliothekskatalogen, Ratgebern und Bibliografien aus Büchern.
        \item Inspiziere die Bücher dieser provisorischen Bibliografie, um das Thema eingrenzen und Bücher daraus entfernen zu können.
    \end{enumerate}
    \item Syntopische Lektüre der ausgewählten Bücher
    \begin{enumerate}
        \item Finde und lies die für dein Problem relevanten Passagen in den ausgewählten Büchern.
        \item Erarbeite dir eine eigene Terminologie, um die verschiedenen Terminologien der Autoren damit fassen zu können.
        \item Formuliere deine eigenen Thesen indem du Fragen erarbeitest und nach den Antworten der Autoren darauf suchst.
        \item Ermittle die Streitfälle, indem du nach Widersprüchen in den Antworten der Autoren suchst.
        \item Untersuche die Diskussion, indem du darlegst, welcher Autor welche Fragen wie beantwortet und warum er sie so beantwortet.
    \end{enumerate}
\end{enumerate}

\bibliographystyle{apacite}
\bibliography{quellen.bib}

\end{document}
