\documentclass[a4paper,12pt]{scrartcl}

\usepackage[ngerman]{babel}
\usepackage{apacite}

\usepackage{chngcntr}
    \counterwithout{subsection}{section}

\usepackage{fontspec}
    \setmainfont{EB Garamond}
    \setsansfont{Open Sans}

\usepackage{setspace}
    \onehalfspace

\begin{document}

\title{How to Read a Book\\\cite{adler-vandooren}}
\subtitle{Zusammenfassung}
\author{Patrick Bucher}
\date{\today}
\maketitle

\tableofcontents

\newpage
\section*{Teil I: Die Dimensionen des Lesens}
\addcontentsline{toc}{section}{Teil I: Die Dimensionen des Lesens}

\subsection{Die Aktivität und Kunst des Lesens}

Wie ein Buch mehr oder weniger aktiv gelesen werden kann, so kann es auch mehr oder weniger gut verstanden werden. Ein guter Leser schöpft mehr Verständnis – und nicht nur Informationen – aus einem Buch als ein schlechter Leser. Zwar kann ein Leser ein Buch zum reinen Informationsgewinn lesen, sofern er dessen Inhalt auf Anhieb vollständig versteht. Dazu müssen Leser und Autor was ihr Verständnis betrifft auf gleicher Stufe stehen. Nur wenn der Leser ein Buch auf Anhieb nicht vollständig versteht, kann dessen Lektüre zu seinem Verständnis beitragen. Zur Erlangung des zusätzlichen Verständnisses kann der Leser entweder ein anderes Buch oder eine andere Person zu Rate ziehen – oder aber sich ohne die Anleitung eines anderen, also durch reines Nachdenken, auf die gleiche Verständnisstufe mit dem Autor bringen. Dies bezeichnet man als gutes Lesen. Je besser ein Leser ist, desto mehr Verständnis kann er aus einem ihm bis dato schwer verständlichen Buch schöpfen, sofern es sich dabei um ein gutes Buch handelt – ein Buch, das ihm neues Verständnis bringen kann.

\subsection{Die Stufen des Lesens}

Es gibt vier Stufen des Lesens, wobei jede Stufe die jeweils tiefere(n) Stufe(n) beinhaltet. Bei der ersten Stufe, dem elementaren Lesen, welches in der Schule eingeübt wird, geht es um das Entziffern von Wörtern und Sätzen. Auf der zweiten Stufe, dem inspizierenden Lesen, geht es darum, sich in kürzester Zeit einen Überblick über ein Buch zu verschaffen. Die dritte Stufe, das analytische Lesen, umfasst das Durchlesen des Buches zur Aufnahme dessen Inhalts. Gründliches Verständnis ist hierbei das Ziel, darum kann für diese Stufe beliebig viel Zeit aufgewendet werden. Auf der vierten Stufe, dem syntopischen Lesen, werden mehrere Bücher eines Themengebiets vergleichend gelesen, um so zu einem höheren, über den einzelnen Büchern stehenden Verständnis zu gelangen. Geübte Leser führen das inspizierende Lesen (Stufe zwei) und das analytische Lesen (Stufe drei) im gleichen Arbeitsschritt durch.

\subsection{Die erste Stufe des Lesens: Elementares Lesen}

Schüler werden nur während der ersten sechs bis neun Schuljahre im Lesen unterwiesen. Sie sind dann funktional lesefähig. Dies bedeutet, dass sie zum Zweck der Informationsaufnahme lesen, aber nicht, dass sie viel Verständnis aus anspruchsvollen Büchern schöpfen können. Darum muss das Lesen, das über die elementare Stufe hinausgeht, auch nach den Pflichtschuljahren weiter eingeübt werden.

\subsection{Die zweite Stufe des Lesens: Inspizierendes Lesen}

Im inspizierenden Lesen geht es darum, in wenigen Minuten bis wenigen Stunden herauszufinden, ob es sich überhaupt lohnt, ein Buch einer analytischen Lektüre (Stufe drei) zu unterziehen. Der erste Schritt des inspizierenden Lesens besteht darin, sich die Titelseite, das Inhaltsverzeichnis, den Umschlagtext und den Index anzuschauen, um so die Struktur des Buches kennenzulernen, und zu erfahren, wie das Buch eingeordnet werden kann. Der Leser sollte im Index nach Begriffen suchen, die ihn im Zusammenhang mit dem Thema des Buches interessieren und die jeweils referenzierten Passagen überfliegen. Oft ist es lohnenswert, die letzten zwei bis drei Seiten des Hauptteils zu überfliegen, wo viele Autoren den Inhalt des Buches rekapitulieren. Auch Vorwort und Einleitung geben oftmals Auskunft über den Aufbau des Buches.

Hat der Leser die Struktur eines Buches erfasst, soll er es im zweiten Schritt oberflächlich durchlesen. Da er bei einem anspruchsvollen Buch ohnehin bei der ersten Lektüre nicht alles verstehen wird, soll er sich zunächst auf das konzentrieren, was er bereits versteht. Ansonsten verzettelt er sich, wodurch seine Motivation schwindet. Die Lücken um das Verstandene herum werden dann beim analytischen Lesen (Stufe 3) geschlossen.

Bücher sollten nicht so schnell als möglich (Speed-Reading), sondern in einem ihnen angemessenen Tempo gelesen werden.

\subsection{Was einen anspruchsvollen Leser ausmacht}

Der anspruchsvolle Leser stellt dem Buch Fragen – und versucht sich diese mittels seiner Lektüre selber zu beantworten. Diese Fragen sind:

\begin{enumerate}
    \item Worum geht es in einem Buch als Ganzes?
    \item Worum geht es in einem Buch im Detail?
    \item Ist das Buch – im Ganzen, im Detail – wahr?
    \item Was soll das Ganze?
\end{enumerate}

Es genügt nicht, die Ansichten des Autors zu kennen. Der anspruchsvolle Leser muss auch eigene Ansichten zum Thema des Buches erarbeiten. Das Buch mit einem Stift zu lesen – Wörter, Sätze und Passagen zu markieren, Notizen zu machen – hilft dabei, aktiv zu lesen. Ein Buch gehört einem erst dann, wenn man sich nicht nur dessen physischen Besitz, sondern auch dessen Inhalt angeeignet hat. Notizen können struktureller Natur (inspizierendes Lesen), konzeptueller Natur (analytisches Lesen) oder dialektischer Natur (syntopisches Lesen) sein.

\newpage
\section*{Teil II: Die  dritte Stufe des Lesens: Analytisches Lesen}
\addcontentsline{toc}{section}{Teil II: Die  dritte Stufe des Lesens: Analytisches Lesen}

\subsection{Zur Einordnung eines Buches}

Der Leser sollte sich immer dessen bewusst sein, was für ein Buch er vor sich hat. Auf erster Stufe unterscheidet man zwischen imaginativer Literatur und Sachbuch. Sachbücher lassen sich in theoretische und praktische Bücher einteilen. Das Ziel eines theoretischen Buches ist es zu vermitteln, was und wie etwas ist – also Verständnis zu befördern: es geht ums Denken. Praktische Bücher beschreiben, wie etwas gemacht werden soll und kann: es geht ums Handeln. Ein Paradebeispiel zur Unterscheidung in theoretische und praktische Werke sind Kants erste beiden Kritiken: Die Kritik der reinen Vernunft (Metaphysik) ist ein theoretisches Buch; die Kritik der praktischen Vernunft (Moral) ein praktisches Buch.

Theoretische Bücher lassen sich in folgende drei Kategorien einteilen: 1) historische Bücher, in denen es um zurückliegende Ereignisse an bestimmten Orten zu bestimmten Zeiten geht; 2) philosophische Bücher, deren Betrachtungen innerhalb der Alltagserfahrung des Menschen angesiedelt ist; und 3) (natur)wissenschaftliche Bücher, bei denen es um bestimmte aber allgemeine Situationen geht, die sich der Alltagserfahrung des Menschen entziehen.

\subsection{Ein Buch «röntgen»}

Die Frage, worum es in einem Buch gehe, kann umformuliert werden in: «Welche Absichten verfolgt der Autor?» Wie jedes Kunstwerk hat auch jedes Buch eine Einheit. Wer nach der Lektüre eines Buches in wenigen Worten oder Sätzen sagen kann, worum es darin geht, der hat diese Einheit erfasst. Weiter kann der Leser sich einen Überblick über die Struktur (das «Skelett») eines Buches verschaffen, indem er die darin behandelten Themen und Unterthemen auflistet und beschreibt. Die Struktur dieser Auflistung kann von derjenigen des Inhaltsverzeichnisses abweichen. Der Leser soll herausfinden, welche Probleme der Autor in seinem Buch behandelt bzw. welche Fragen er darin zu beantworten versucht. Der Autor sollte nicht etwa einer Psychotherapie unterzogen werden; der Leser muss sich innerhalb des Abgedruckten orientieren.

\subsection{Sich mit dem Autor auf Begriffe einigen}

Der Leser muss sich mit dem Autor auf Begriffe «einigen», d.h. zu verstehen versuchen, welche Bedeutung(en) der Autor bestimmten Wörtern beimisst. Ein Wort kann für mehrere Begriffe stehen; ein Begriff kann durch mehrere Worte bezeichnet sein: zwischen Wort und Begriff liegt eine m-zu-n-Beziehung vor. Begriffe sind Wörter, die ein Autor mit Bedeutung(en) «auflädt». Es gehört zur Beherrschung der Kunst des Lesens, diese Begriffe und ihre besondere Bedeutung zu ermitteln, die sie für den Autor haben. Solche Wörter sind teils explizit mit typografischen Mitteln oder als Definitionen hervorgehoben – oder der Autor grenzt seinen Gebrauch eines Begriffs gegen dessen Verwendungen seiner Kollegen ab. Autoren verwenden auch Fachvokabular oder gar ein Privatvokabular (etwa der Seinsbegriff im Existenzialiusmus oder der Verstehensbegriff in der Hermeneutik). Wörter werden unter dem Begriff Vokabular, Begriffe unter dem Begriff Terminologie subsumiert. Ein Begriff kann mehr oder weniger breit gefasst und durch Synonyme bezeichnet werden.

Um die Bedeutung eines unbekannten Begriffes zu erschliessen, betrachtet der Leser die ihm bereits bekannten Begriffe wie zusammengesetzte Puzzleteile, die zwar nur ein unvollständiges Bild ergeben, das ihm aber dazu genügt, das fehlende Puzzleteil zu erraten: den unbekannten Begriff.

\subsection{Die Aussage des Autors erfassen}

Nähert sich der Leser beim Erkennen der strukturellen Organisation eines Buches von oben nach unten – also von Kapitel zu Abschnitt, von Abschnitt zu Absatz, von Absatz zu Satz und schliesslich von Satz zu Wort –, geht er bei der Interpretation in entgegengesetzter Richtung vor, also von unten nach oben: von Begriff zu These, von These zu Argument. Aus den Wörtern schöpft der Leser Begriffe, aus Sätzen Thesen und aus den Beziehungen der Sätze untereinander Argumente. Thesen sind in Schlüsselsätzen zu finden, die teils typografisch hervorgehoben sind. Ist dies nicht der Fall, muss sich der Leser selbst zu helfen wissen: wichtige Sätze sind solche, die sein Verständnis erweitern können, und ihm somit nicht nach dem ersten Durchlesen völlig verständlich sind. Sie enthalten oftmals Schlüsselbegriffe, die der Leser bereits erarbeitet hat. In der Regel sind wichtige Sätze Prämissen oder Folgerungen. Um zu überprüfen, ob der Leser eine These oder eine Argumentation verstanden hat, soll er diese in eigene Worte zu fassen versuchen. Bei der Lektüre eines fremdsprachigen Buches ist auch die freie Übersetzung eines solchen Schlüsselsatzes in die eigene Sprache eine gute Verständnisprüfung. Wer eine These verstanden hat, muss auch ein Beispiel dazu machen können.

\subsection{Ein Buch fair kritisieren}

Um ein Buch kritisieren – d.h. bewerten! – zu können, muss der Leser es zuerst verstanden haben. Es ist wichtig, das Argument eines Autors zu Ende zu verfolgen, bevor eine Wertung darüber abgegeben wird. Vor und während der Lektüre muss der Leser darauf vorbereitet sein, dass er mit dem Autor einverstanden oder nicht einverstanden sein wird. Oder er muss mit der Bewertung zuwarten, da er noch nicht über das dazu notwendige Wissen und Verständnis verfügt. Zwischen persönlicher Meinung und begründeter Ansicht besteht ein gewaltiger Unterschied. Meinungsverschiedenheiten, die von rationalen Argumenten gestützt werden, können grundsätzlich überwunden werden.

\subsection{Einem Autor zustimmen oder widersprechen}

Für eine faire Bewertung eines Buches muss es nicht «nur» verstanden worden sein, es erfordert auch weitere zwingende Vorbedingungen: 1) Der Leser muss sich dessen bewusst sein, dass Menschen von Gedanken und von Gefühlen geleitet werden, letztere für die Bewertung jedoch nicht ausschlaggebend sein dürfen. 2) Der Leser muss die Prämissen des Autors zu akzeptieren versuchen, um dessen Argumentation nachvollziehen und bewerten zu können. 3) Der Leser muss ein Buch wohlwollend lesen, d.h. versuchen, den Gesichtspunkt des Autors einzunehmen.

Es gibt vier Gründe, mit einem Autor nicht zufrieden zu sein: er ist in seinem Buch 1) uninformiert, 2) fehlinformiert, 3) unlogisch oder 4) unvollständig. Die ersten beiden Punkte hängen insofern zusammen, als dass der Autor nicht über die Wahrheit verfügt. Man beachte, ob das Wissen zum Zeitpunkt der Niederschrift schon vorhanden war! (Es wäre unfair, Darwin für die Unkenntnis der genetischen Vererbungslehre zu rügen, die erst viel später von Mendel entdeckt wurde.) Unlogisch ist die Argumentation, wenn aus (wahren oder falschen) Prämissen die falschen Schlüsse gezogen werden, d.h. der Autor die Regeln der formalen Logik nicht einhält. (Beispiel: Alle Affen sind Säugetiere, ergo sind alle Säugetiere Affen.) Eine unlogische Argumentation liegt auch dann vor, wenn diese inkonsistent ist, d.h. Widersprüche enthält. Das Werk ist unvollständig, wenn der Autor nicht das hält, was er eingangs versprochen hat: er beantwortet die von ihm gestellte Frage nicht (zureichend) – oder er hat nicht alles aus seinem Material herausgeholt. Bei den Punkten eins bis drei widerspricht der Leser zu einem gewissen Grad dem Autor, bei Punkt vier ist er zwar mit ihm einverstanden, jedoch noch nicht zufrieden.

\subsection{Hilfsmittel zum Lesen}

Analytische Lektüre ist intrinsische Lektüre. Der Leser geht dabei nicht über das zu lesende Buch hinaus. Bedient er sich weiteren Wissens, das nicht im Buch vorhanden ist, spricht man von extrinsischer Lektüre. Extrinsische Hilfsmittel sind:

\begin{description}
\item[Erfahrungen] – allgemeiner oder spezieller Natur. Erstere sind wichtig für die Lektüre von Philosophie und imaginativer Literatur und umfassen die Alltagserfahrungen des Menschen; zweitere umfassen aus Forschung gewonnene (empirische) Erkenntnisse, die nicht jedem zur Verfügung stehen. Solche Erfahrungen sind für die Lektüre (natur)wissenschaftlicher Werke notwendig. Für historische Werke braucht der Leser aus beiden Erfahrungsschätzen zu schöpfen.
\item[Bücher] Die meisten Autoren sind auch Leser anderer Bücher. Darum kann kaum ein Buch isoliert betrachtet werden. Wichtig ist die chronologische Reihenfolge des Erscheinens einzelner, teils aufeinander aufbauender Bücher zu einem Wissensgebiet.
\item[Kommentare] und Lektürehilfen sollten erst nach der Lektüre des Buches zuhilfe gezogen werden, da der Leser sonst in eine bestimmte Interpretationsrichtung gedrängt werden und dann das Buch nicht mehr unvoreingenommen lesen könnte. Kommentare sind zudem – wie jedes Buch – möglicherweise fehlerhaft und unvollständig. Inhaltsangaben können dem Leser dabei helfen, den Inhalt eines Buches aufzufrischen. Abstracts dienen als Entscheidungshilfe, ob ein (zusätzliches) Buch überhaupt gelesen werden soll.
\item[Referenzen] verlangen vom Leser a) dass er weiss, was er erfahren will – welche Wissenslücke will er durch die Konsultation stopfen? b) Kenntnis über die verschiedenen Arten von Referenzwerken im Allgemeinen und c) Kenntnis über das zu konsultierende Referenzwerk im Besonderen. Der Leser muss auch wissen, welche Arten von Fragen überhaupt mit Referenzwerken beantwortet werden können.
\item[Wörterbücher] klären über Rechtschreibung, Aussprache, Bedeutungen, Gebrauch und Herkunft von Wörtern auf.
\item[Enzyklopädien] liefern dem Leser «harte» Fakten über Sachverhalte, über die sich die Menschheit weitgehendst einig ist.Über Fakten sollte nicht gestritten werden – man sollte sie besser nachschlagen.
\end{description}

\newpage
\section*{Teil III: Ansätze für verschiedene Arten von Büchern}
\addcontentsline{toc}{section}{Teil III: Ansätze für verschiedene Arten von Büchern}

\subsection{Wie man praktische Bücher liest}

Je weiter Regeln gefasst werden, desto weiter sind sie von ihrem Anwendungsgegenstand entfernt. Unterschiedliche Arten von Büchern erfordern präzisierte Regeln. Gerade imaginative Literatur lässt sich mit den bisher geschilderten Regeln kaum fassen.

Auch für das Lesen praktischer Bücher gelten besondere Regeln. Im Gegensatz zu theoretischen Büchern, die ein theoretisches Problem formulieren und zu lösen versuchen, lösen praktische Bücher selber keine Probleme, sondern bieten dem Leser Handlungsanweisungen, damit er damit Probleme lösen kann. Es gibt zwei Arten von praktischen Büchern, wobei der Übergang fliessend ist: 1) Bücher, die Regeln und Handlungsweisen darlegen, und 2) Bücher, die sich mit den Prinzipien befassen, aus denen Regeln abgeleitet werden. Die Thesen praktischer Bücher sind oft die Regeln selbst, die man anhand ihres imperativen Aussagemodus erkennt. Die Argumentation umfasst dann die Begründung dieser Regeln.

Für die Bewertung praktischer Bücher gibt es zwei Kriterien: 1) die Handlungsanweisungen funktionieren, und 2) sie führen zu einem erwünschten Ziel. Das Ziel fungiert hier als Prämisse, mit der ein Leser möglicherweise nicht einverstanden ist. Schliesslich kann der Leser andere Ziele verfolgen, d.h. die vom Autor beschriebenen Absichten nicht gutheissen. Bei der Lektüre von praktischen Büchern muss sich der Leser zwei Fragen stellen: 1) Welche Ziele gibt der Autor vor? 2) Welche Mittel bietet er zu deren Erreichung an? In einem praktischen Buch versucht der Autor immer den Leser zu überzeugen. Dazu bedient er sich auch Rhetorik, die auf Gefühle, und nicht auf den Verstand abzielt. Solche Rhetorik ist nur dann gefährlich, wenn der Leser sie nicht als solche erkennt. Es ist von Vorteil, die Hintergründe und Lebensumstände des Autors zu kennen.

Die Frage, welche Probleme der Autor zu lösen versucht (Regel 4 des analytischen Lesens), kann umformuliert werden in: «Wozu möchte mich der Autor verleiten, welche Ziele verfolgt er mit mir?» Die Frage, ob der Autor sein Problem gelöst hat (Regel 8 des analytischen Lesens), lautet dann folglich: «Welche Mittel gibt mir der Autor an die Hand und sind diese hinreichend?» Die Frage, was denn das alles soll (Frage 4 des aktiven Lesens), wird dahingehend beantwortet, dass der Leser seine Ansichten über einen Sachverhalt ändert und seine Handlungsweise entsprechend anpasst.

\subsection{Wie man imaginative Literatur liest}

Das Lesen imaginativer Literatur erfordert eine Anpassung der bisher geschilderten Regeln. Imaginative Literatur vermittelt nicht Wissen, sondern Erlebnisse. Sie erfordert nicht Urteilsvermögen, sondern Vorstellungskraft. Damit sich die Vorstellung des Lesers entfalten kann, darf er der Wirkung, die das Werk auf ihn ausübt, nicht zu widerstehen versuchen. Imaginative Literatur lebt von sprachlichen Ambiguitäten: Zwischen den Zeilen steht mindestens so viel wie in den Zeilen. Die Aussage eines imaginativen Werkes besteht nicht im abgedruckten Text allein. Imaginative Literatur erfordert einen angepassten Wahrheitsbegriff: Das Geschilderte muss nicht wahr sein im Sinne des faktisch Überprüfbaren, es muss plausibel sein in der Welt, die der Autor errichtet.

Ein imaginatives Werk lässt sich einordnen (Roman, Theaterstück usw.) und hat eine Einheit: den Plot, der sich in wenigen Sätzen als Inhaltsangabe beschreiben lässt. Er umfasst das, was in den Zeilen, nicht zwischen den Zeilen steht. Die Handlung imaginativer Werke ist nicht logisch-strukturell angeordnet, sondern chronologisch in eine bestimmte (nicht zwingend lineare) Ordnung gebracht. Statt Begriffe ermittelt der Leser Episoden, Vorfälle, Charaktere, Gedanken, Reden, Gefühle und Handlungen. Statt Thesen aufzuspüren muss sich der Leser in die geschilderte Welt hineinversetzen. Statt einer Argumentation entwickelt sich eine Handlung.

Auch der Leser eines imaginativen Werkes darf dieses erst dann kritisieren, wenn er es verstanden hat. Dazu muss er zumindest versuchen, das Geschilderte nachzuempfinden. Wie er in einem Sachbuch die Prämissen akzeptieren muss, um die Argumentation nachzuvollziehen, muss er in einem imaginativen Werk die Welt hinnehmen, wie sie ihm der Autor vorlegt, damit er die Handlung nachvollziehen kann. Bei imaginativer Literatur lautet das Urteil nicht auf «einverstanden»/«nicht einverstanden», sondern auf «gefällt mir»/«gefällt mir nicht», wobei Zustimmung und Ablehnung begründet werden muss.

\subsection{Vorschläge für das Lesen von Prosa, Theaterstücken und Gedichten}
\subsection{Wie man Geschichte liest}
\subsection{Wie man Naturwissenschaft und Mathematik liest}
\subsection{Wie man Philosophie liest}
\subsection{Wie man Sozialwissenschaften liest}

\newpage
\section*{Teil IV: Die höchsten Ziele des Lesens}
\addcontentsline{toc}{section}{Teil IV: Die höchsten Ziele des Lesens}
\subsection{Die vierte Stufe des Lesens: Syntopisches Lesen}
\subsection{Lesen und das Wachsen des Verstandes}

\newpage
\section*{Die Regeln im Überblick}
\addcontentsline{toc}{section}{Die Regeln im Überblick}
\subsection{Fragen für das aktive Lesen}
\subsection{Regeln für das inspizierende Lesen}
\subsection{Regeln für das analytische Lesen}
\subsection{Stufen des syntopischen Lesens}

\newpage
\bibliographystyle{apacite}
\bibliography{quellen.bib}

\end{document}
